\chapter{Discussion and Conclusion}
\label{sec:conclusions} % Always give a unique label

\section{Discussion}
\subsection{MRI Experiment}
The experimental results show that key components of closed-loop model-guided needle localization all work in conjunction. The needle artifact centroids are all correctly identified by thresholding the image and identifying the region with the centroid closest to the estimated needle position in the scan. Needle cross-section identification finds the correct artifact even when other large non-needle artifacts are present.

The error relative to the artifact cross-section centroid outside the tip artifact region is less than 0.5mm, which is comparable to other work in needle localization in US\cite{rossa_adaptive_2016} and MRI\cite{song_biopsy_2012}. The length constraint allows the needle curve to extend beyond the furthest sampled point into the tip artifact region, but the error is high in the tip artifact region because the artifact is lopsided and its centroid is not located on the needle shaft. This is a shortcoming in the baseline dataset and does not reflect the error relative to the actual position of the needle. Extending the dataset to include a CT scan at each needle insertion step would provide a superior baseline for the entire needle.

The linear model exhibits significant error relative to the baseline and misrepresents the shape of the needle for a majority of its length. Choosing a polynomial with a degree greater than $n=3$ does not produce a significant reduction in error since the needle trajectory in the MRI dataset deflects only in one direction.

Increasing the number of observation points along the length of the needle does not reduce the error relative to the baseline. It is more imporant that the observation points be spaced evenly along the portion of the needle visible in imaging.

In general, the concept of modeling the needle shape by sampling needle cross-sections in MRI is sound and could be extended for real-time applications.

\subsection{Needle Model}
The bending energy model is able to produce a good fit for the needle with very few sampled points. It does not over-fit, even when the small number of sample points would otherwise underconstrain the model. The curve between sample points approximately matches the actual position of the needle, and the estimated shape of the needle lies within the artifact region.

Optimization takes between 5 and 10 seconds (Lenovo ThinkPad P50, Intel Xeon CPU E3-1505M v5 @ 2.80GHz, 16 GB RAM) to converge to a solution depending on the constraints. The constraints on the needle length and the average error contribute to increased processing time. This is not an obstacle for an offline experiment, but it would present issues for real-time imaging. The choice of optimization algorithm likely has a significant impact on the total processing time. SLSQP performs the optimization sequentially, and an NLP optimizer designed to take advantage of parallel processing would probably complete the computation in a much shorter time.

The bending energy minimization approach is not guaranteed to provide a feasible solution for every possible combination of constraints an sample points. Even if a solution is found, it is also not guaranteed to finish the optimization within a constant time, which might complicate integration into a real-time system.


\section{Future Work}
An important pice of follow-on work will be to demonstrate real-time tracking using live MR imaging. This will require implementation of a communication protocol that transmits scan plane poses to the MRI controller and listens for new image data. Precedent exists for controlling an MRI scanner in this way\cite{patel_closed-loop_2015}.

The time required to compute the needle curve optimization is very high and not currently suited for real-time operation. Possible solutions to reduce the computational load include reducing the number of numerical approximations in the optimization function, choosing a more efficient NLP optimization algorithm, and rewriting the needle modeling Python packages in C++.

\section{Conclusion}
This thesis presented a closed-loop model-based needle localization strategy agnostic to the imaging modality and independent of tissue mechanical properties. A simulated multi-step needle insertion in MRI was tracked, and the error between the estimated position of the needle shaft from the model and the measured position of the centroid of the needle artifact was comparable to previously-published needle localization approaches. The bending energy minimization approach produces accurate curve fits using a small number of  images by also considering the kinematics of the insertion platform. While not extensively explored in this work, the parametric polynomial needle curve and the concept of planning scan planes using the curve could work for a very wide array of needle trajectories, including loops and other paths rarely explored in other literature.
