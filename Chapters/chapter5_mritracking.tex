%%%%%%%%%%%%%%%%%%%%% chapter.tex %%%%%%%%%%%%%%%%%%%%%%%%%%%%%%%%%
%
% sample chapter
%
% Use this file as a template for your own input.
%
%%%%%%%%%%%%%%%%%%%%%%%% Springer-Verlag %%%%%%%%%%%%%%%%%%%%%%%%%%

%\begin{savequote}[8cm]
%  ``Veni, vidi, vici.''
%  \qauthor{Julius Caesar}
%\end{savequote}


\chapter{Needle Localization in MR Images}
% \label{stereotracking} % Always give a unique label
% use \chaptermark{}
% to alter or adjust the chapter heading in the running head

\section{MRI Dataset}
- Talk about scan sequence that was used: 3D Fast Field Echo in a 3T MRI at UMass Medical Center

- Biopsy needle inserted into agar gelatin phantom.

- 11.9mm steps, determined by removable spacers

- Plastic needle mount attached to phantom keeps needle in a known orientation, so needle base pose can be calculated for each scan volume.

\section{Software Architecture}
\subsection{Simulated MRI Scanner}
- 3D Slicer module using an existing volume.

- Return a slice of the data at a specified scan plane pose and field of view for a given time.

\subsection{Needle Tracking Module}
- 3D Slicer module

- Callback function registed to MRML node associated with MRI data

- When node is updated, finds position of needle tip in the scan.

\section{Experiment}
- Use data from two prior insertions

- Ground truth is error in each axial scan plane between the modeled needle curve and the centroid of needle artifact cross-section.

- Models under comparison are a line and a 2nd-degree polynomial fit by least squares to the sampled points, and the needle mechanical model.

- Needle base kinematics set manually for each step of insertion.


\section{Results and Discussion}
- Add plot for in-plane error between ground truth and curve as function of insertion depth, for each set of scans

- Linear fit isn't accurate, since needle does deflect. Misrepresents tip pose.

- Low-order polynomial is an OK fit.

- Mechanical model is a good fit, even with fewer sampled points.

- Experiment was conducted offline, so the increased processing time for fitting mechanical model is OK.
