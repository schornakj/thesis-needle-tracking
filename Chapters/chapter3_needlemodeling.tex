%%%%%%%%%%%%%%%%%%%%% chapter.tex %%%%%%%%%%%%%%%%%%%%%%%%%%%%%%%%%
%
% sample chapter
%
% Use this file as a template for your own input.
%
%%%%%%%%%%%%%%%%%%%%%%%% Springer-Verlag %%%%%%%%%%%%%%%%%%%%%%%%%%

%\begin{savequote}[8cm]
%  ``Veni, vidi, vici.''
%  \qauthor{Julius Caesar}
%\end{savequote}


\chapter{Needle Modeling}
\label{needlemodel} % Always give a unique label
% use \chaptermark{}
% to alter or adjust the chapter heading in the running head

- Linear finite element model doesn't hold for a thin flexible needle

- Don't have enough information from observations to get all displacements and forces to get a nonlinear FE model to work

- Kinematic model is hard to update from measurements

\section{Beam Bending Energy}

The transverse bending energy in a beam is a function of the curvature in the beam.

\begin{equation}
\label{eq:bending_energy_transverse}
U_B = \frac{EI}{2}\int_{0}^{l_i}\frac{1}{\rho^2_i}dx
\end{equation}

% \begin{equation}
% \label{eq:bending_energy_axial}
% U_B = \frac{EI}{2}\int_{0}^{l_i}\frac{1}{\rho^2_i}dx
% \end{equation}

\begin{equation}
\label{eq:curvature}
\frac{1}{\rho_i} = \frac{d^2v_i/dx^2}{(1+(dv_i/dx)^2)^{3/2}} ~= \frac{d^2v_i}{dx^2}
\end{equation}

The actual biopsy needle contains several parts made of different materials, including an inner rod that sides within the outer shell. As a simplification, the needle is assumed to be solid and cylindrical, so its area moment of inertia can be written as \ref{eq:beam_inertia}. The change in the cross-sectional shape of the needle at its tip is neglected.

\begin{equation}
\label{eq:beam_inertia}
I = \frac{\pi}{4}r^4
\end{equation}

\section{Parametric Polynomial Space Curves}
The needle curve is represented using the \textit{n}-degree parametric polynomial function shown in Equation \ref{eq:parametric_curve}. 

\begin{equation}
\label{eq:parametric_curve}
\begin{cases} x(t) = a_n t^n + a_{n-1} t^{n-1} + ... + a_1 t + a_0 \\
 y(t) = b_n t^n + b_{n-1} t^{n-1} + ... + b_1 t + b_0  \\
 z(t) = c_n t^n + c_{n-1} t^{n-1} + ... + c_1 t + c_0 \end{cases} \\
 t\in (0, 1)
 \end{equation}

The three spatial coordinates \textit{x}, \textit{y}, and \textit{z} are functions of a parameter \textit{t}, which ranges from 0 at the needle base to 1 at the needle tip. While an alternative implementation could represent two components of the coordinate as a function of the third, using an independent parameter allows the curve to represent torturous trajectories without restricting available directions of needle insertion.

The maximum number of inflection points, and consequently the maximum number of changes in needle direction that the curve can represent, is limited by the degree of the polynomial.

\section{Curve Fitting}
The purpose of curve fitting is to choose coefficients of the parametric function in Equation \ref{eq:parametric_curve} given a number of observed needle cross section coordinates so that the total bending energy in the curve and the error between the curve and the needle coordinates are minimized.

Prior to optimization, the initial coefficients for each curve are found using a least-squares polynomial for the needle coordinates. This generates a solution close to the minimum bending energy curve, but it is not representative of the actual mechanical factors that determine the shape of the needle.

The curve is optimized to minimize bending energy using Sequential Least SQuares Programming (SLSQP), which is an iterative quadratic programming search algorithm for nonlinear equations. A detailed explanation of this algorithm can be found in Kraft.

\subsection{Cost Function}
The cost function to be minimized is an extension of Equation \ref{eq:bending_energy_transverse} with the addition of the weighted total error between the needle coordinates and the nearest point on the curve. 

While equality constraints can also be used to guide the optimized curve to intersect all the needle coordinates, this approach risks over-constraining the curve for lower-degree polynomials.

\begin{equation}
\label{eq:cost_function}
U_B = \frac{EI}{2}\int_{0}^{l_i}\frac{1}{\rho^2_i}dx + \kappa\sum errors_{current}
\end{equation}

\subsection{Slope Constraints}
The SLSQP optimization is constrained so the curve is tangential to the direction of insertion at the needle entry point.

\begin{equation}
\label{eq:slope_constraint_y}
\frac{dy(t_{entry})}{dx(t_{entry})} = orientation_{entry,y}
\end{equation}

\begin{equation}
\label{eq:slope_constraint_z}
\frac{dz(t_{entry})}{dx(t_{entry})} = orientation_{entry,z}
\end{equation}

\subsection{Length Constraint}
The optimization is further constrained such that the length of the curve between t=0 and t=1 is equal to the length of the needle.

\begin{equation}
\label{eq:length_constraint}
L_{needle} = \int_0^1 \sqrt[]{\frac{dx(\tau)}{dt}^2 + \frac{dy(\tau)}{dt}^2 + \frac{dz(\tau)}{dt}^2} d\tau
\end{equation}

\subsection{Assumptions}
The selection of values to associate between \textit{t} and the needle coordinates assumes that length of the needle model curve is close to the magnitude of the vector between points on the needle. This implies that the deflection of the needle tip throughout insertion will be comparatively small.

\section{Piecewise Polynomial Space Curves}
There is a practical upper limit on the degree of the polynomial imposed by the regression and optimization function, which means that some needle trajectories cannot be practically represented using a single polynomial. In these cases a piecewise polynomial function can be used, with additional constraints imposed so that the curves are contiguous and continuous at the points of intersection.

\section{Needle Motion Model}
- Bending model provides a reasonable estimation of the pose of the needle tip throughout insertion.

- Still need to know where the needle is going to go during insertion, to plan future imaging

- Can draw a short straight line aligned with the pose of the needle, which assumes that any curvature will be small enough that the cross section of the needle could be captured in the imaging window.

- Could also use the kinematic bicycle model to find an offset tip position using the needle base orientation.